\documentclass{article}
\usepackage{listings}
\usepackage{geometry}

\geometry{
    a4paper,
    total={170mm,257mm},
    left=20mm,
    top=20mm,
    bottom=20mm,
    heightrounded
}

\begin{document}
\header{Report on the Strength of Concrete}
\section{CatBoost}
\align{
    \begin{lstlisting}[language=Python,basicstyle=\ttfamily]
    Best hyperparameters: {'depth': 7, 'iterations': 1000, 
    'learning_rate': 0.05}
    Expected RMSE error: 4.081210766971801
    MAE: 2.5745905532392674
    MSE: 16.656281324446557
    R2: 0.935359818094655
    95% confidence interval:
\left(\left[51.8243495 , 41.93664799, \dots, 42.298315...\right]\right)
\end{lstlisting}
}

\section{Support Vector Machines}
\align{
    \begin{lstlisting}[language=Python,basicstyle=\ttfamily]
    Best hyperparameters: {'C': 1000, 'coef0': 0, 'degree': 3,
    'gamma': 0.1}
    0.8907982261640797
    SVC(C=1000, coef0=0, gamma=0.1, random_state=42)
    [[96 6]
    [14 90]]
    Expected RMSE error: 0.3115884764248779
    MAE: 0.0970873786407767
    MSE: 0.0970873786407767
    R2: 0.6116138763197587
    95% confidence interval: (0.397897447289056, 0.5341413876624003)
\end{lstlisting}
}


\section{Random Forests}
\align{
    \begin{lstlisting}[language=Python,basicstyle=\ttfamily]
    Best hyperparameters: {'max_depth': 10,'min_samples_leaf':1,
    'min_samples_split':2, 'n_estimators': 500}
    -26.80270840004423
    RandomForestRegressor(max_depth=10, n_estimators=500, random_state=42)
    MAE: 3.736334329056867
    MSE: 29.854417119395947
    RMSE: 5.463919574755465
    R2: 0.8841401081258502
    Start: 6.579025407891052
    End: 62.99933535147096
\end{lstlisting}
}

\section{AdaBoost}
\align{
    \begin{lstlisting}[language=Python,basicstyle=\ttfamily]
    Best hyperparameters: {'n_estimators': 200, 'loss': 'square', 
    'learning_rate': 1}
    -55.76729644915056
    AdaBoostRegressor(learning_rate=1, loss='square', 
    n_estimators=200,
                      random_state=42)
    Mean Absolute Error: 6.270046832695762
    Mean Squared Error: 57.745066862245814
    Root Mean Squared Error: 7.599017493218832
    R2 Score: 0.7759012619081165
    95% confidence interval:
\left(\left[39.08921757, 37.82768758, \dots, 40.37628906...\right]\right)
\end{lstlisting}
}

\section{Conclusion}
\align{
    \begin{verbatim}
        The best model that was tested was the CatBoost model.
        The model was trained on 80 percent of the data and tested 
        on the remaining 20 percent. The reason why the model
        performed the best was it produced,the Expected RMSE error:
        4.081210766971801, MAE: 2.5745905532392674, MSE: 16.656281324446557,
        R2: 0.935359818094655. Since the mode produced an RMSE that was 
        low, it means that the predictions that the model produced were
        extremely close to the actual values. Furthermore,The R2 value 
        was rather high, this means that the model was able to explain 
        varience that occured in the data. Finally when looking at the 
        95 percent confidence intervals. They are rather narrow which
        means that the model was able to predict the values with a high 
        degree of accuracy.
    
        Check out the mybest.py to see the speed of the Kfolds as well
        as permutation importance. Which will help your business focus
        on attributes that are important your business and the strength
        of concrete. This will allow you hone your formula to deliver 
        the best product to your customers. 

    \end{verbatim}
}

\end{document}

